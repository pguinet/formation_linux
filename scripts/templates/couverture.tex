% Template de couverture pour les PDFs Formation Linux
% Version LaTeX

\documentclass[12pt,a4paper]{article}
\usepackage[utf8]{inputenc}
\usepackage[french]{babel}
\usepackage{geometry}
\usepackage{graphicx}
\usepackage{xcolor}
\usepackage{tikz}
\usepackage{fancyhdr}
\usepackage{tcolorbox}

% Configuration de la page
\geometry{margin=0cm}
\pagestyle{empty}

% Définition des couleurs
\definecolor{primaryblue}{RGB}{44, 82, 130}
\definecolor{lightgray}{RGB}{245, 245, 245}
\definecolor{darkgray}{RGB}{90, 90, 90}

\begin{document}

% === PAGE DE COUVERTURE ===
\begin{tikzpicture}[remember picture, overlay]
    % Fond avec dégradé
    \fill[primaryblue] (current page.north west) rectangle ([yshift=-8cm]current page.north east);
    \fill[lightgray] ([yshift=-8cm]current page.north west) rectangle (current page.south east);
    
    % Bande décorative
    \fill[darkgray] ([yshift=-8.2cm]current page.north west) rectangle ([yshift=-8.5cm]current page.north east);
\end{tikzpicture}

\vspace{1cm}

% Image de couverture
\begin{center}
    \includegraphics[width=8cm]{ressources/images/Formation_Linux.png}
\end{center}

\vspace{1cm}

% Titre principal
\begin{center}
    {\Huge\color{white}\textbf{FORMATION LINUX}}
    
    \vspace{1cm}
    
    {\LARGE\color{white} $title$ }
    
    \vspace{0.5cm}
    
    {\large\color{white} Guide complet pour maîtriser Linux}
\end{center}

\vspace{2cm}

% Section centrale avec informations
\begin{center}
\begin{tcolorbox}[colback=white, colframe=primaryblue, width=12cm, arc=5mm]
    \centering
    \textbf{\large Public cible}
    
    \vspace{0.3cm}
    
    Débutants et utilisateurs souhaitant\\
    maîtriser les fondamentaux de Linux
    
    \vspace{0.5cm}
    
    \textbf{\large Contenu}
    
    \vspace{0.3cm}
    
    8 modules de formation • Travaux pratiques\\
    Navigation • Fichiers • Droits • Processus\\
    Réseaux • Scripts • Automatisation
\end{tcolorbox}
\end{center}

\vspace{\fill}

% Bas de page
\begin{center}
    \textcolor{darkgray}{\large $date$}
\end{center}

\newpage

% === QUATRIÈME DE COUVERTURE ===
\begin{tikzpicture}[remember picture, overlay]
    % Fond avec dégradé inversé
    \fill[lightgray] (current page.north west) rectangle ([yshift=-16cm]current page.north east);
    \fill[primaryblue] ([yshift=-16cm]current page.north west) rectangle (current page.south east);
    
    % Bande décorative
    \fill[darkgray] ([yshift=-15.8cm]current page.north west) rectangle ([yshift=-16.2cm]current page.north east);
\end{tikzpicture}

\vspace{2cm}

% À propos de l'auteur
\begin{center}
\begin{tcolorbox}[colback=white, colframe=primaryblue, width=13cm, arc=5mm]
    \textbf{\Large À propos de l'auteur}
    
    \vspace{0.5cm}
    
    \textbf{Pascal Guinet} est expert DevOps chez Prima Solutions, leader français de l'AssurTech. Spécialisé dans les infrastructures cloud et l'intégration continue, il accompagne les équipes dans leur transformation digitale.
    
    \vspace{0.3cm}
    
    Fort de plusieurs années d'expérience dans l'écosystème Linux et les technologies cloud, Pascal forme régulièrement des équipes aux bonnes pratiques DevOps et à la maîtrise des systèmes Unix/Linux.
    
    \vspace{0.3cm}
    
    \textit{Prima Solutions - Expert DevOps \& Formateur}
\end{tcolorbox}
\end{center}

\vspace{2cm}

% Licence Creative Commons
\begin{center}
\begin{tcolorbox}[colback=white, colframe=darkgray, width=13cm, arc=3mm]
    \centering
    
    \includegraphics[width=3cm]{ressources/images/licenses/cc-by-nc-sa.png}
    
    \vspace{0.3cm}
    
    \textbf{Licence Creative Commons}\\
    \textit{Attribution - Pas d'Utilisation Commerciale - Partage dans les Mêmes Conditions 4.0}
    
    \vspace{0.2cm}
    
    {\small Ce document peut être partagé et adapté selon les conditions de la licence CC BY-NC-SA 4.0}
\end{tcolorbox}
\end{center}

\vspace{\fill}

% Pied de page avec informations
\begin{center}
\textcolor{white}{\large\textbf{Formation Linux - Prima Solutions}}

\vspace{0.3cm}

\textcolor{white}{www.prima-solutions.com}
\end{center}

\end{document}